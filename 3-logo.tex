\chapter{Logo} \label{cap:papert2}

Questo capitolo è un'introduzione all'uso del logo. Si rivolge principalmente agli studenti di formazione primaria, cercando di mostrare come iniziare i bambini alla pratica del logo. È interessante notare che, avendo proposto questo argomento a diverse centinaia di studenti, mi sono reso conto come insegnare il Logo ai bambini o ai miei studenti sia molto simile. E ancora più come, in media, nel primo approccio i bambini siano più bravi! Ne discutiamo spesso con quegli studenti che affermano di aver avuto grandi difficoltà a parlare con la tartaruga. La maggior parte di loro sperimenta una sorta di "risurrezione" dopo alcune lotte iniziali: "All'inizio non ho capito niente, ma poi... che meraviglia!  Ma alcuni non si liberano mai dal malumore, almeno nelle due settimane e mezza di laboratorio. Parlare con questi studenti è sempre molto interessante perché sono stupiti quando dico loro che, il più delle volte, i bambini di 9 anni vanno d'accordo con la tartaruga abbastanza facilmente. Quando si cerca di scavare in questo paradosso, sembra che i bambini colgano naturalmente l'aspetto ludico della situazione. D'altra parte, gli adulti si lasciano facilmente intrappolare nei loro pregiudizi: non ho una testa per i numeri, io e i computer siamo molto lontani, non sono mai andato d'accordo con le tecnologie. Soprattutto l'ultima è un'affermazione che ho sentito tante volte, come rilevato nel capitolo precedente, e, sorprendentemente, da veri nativi digitali. Cioè, una persona che può avere migliaia di contatti su Facebook, candidamente può poi sostenere di non andare d'accordo con le tecnologie.

Detto questo, ora affrontiamo la tartaruga con lo spirito più giocoso possibile, a partire da zero. Dato che suppongo sappiate bene come i bambini affrontano un nuovo gioco, provate a fare lo stesso.  Quindi, se avete scaricato LibreOffice e attivato la barra degli strumenti Logo, come spiegato nel primo capitolo, aprite un nuovo documento di testo. Vi trovate di fronte a un foglio bianco: qui potete scrivere una lettera, una lista della spesa, una poesia o quello che volete. Provate a scrivere questo\footnote{Uso i caratteri maiuscoli solo per chiarezza, LibreLogo è "insensibile alle maiuscole e minuscole"}:

\vskip 1cm

\begin{scriptsize}
\begin{minipage}{0.45\textwidth}
\begin{itemize}[itemsep=-3pt,parsep=2pt]
\item[] \hspace{0.5cm} FORWARD 100
\end{itemize}
\end{minipage}
\end{scriptsize}

\vskip 1cm

Poi premete il pulsante "Play" nella barra degli strumenti (quello con la freccia verde, ignorate gli altri pulsanti per il momento)
Cosa vedete?  Dimenticate i vostri "limiti" e tutte le opinioni chepotete avere sui vostri rapporti con la matematica, la tecnologia e così via. Guardate il risultato ottenuto e confrontatelo con quello che avete scritto. Ponetevi delle domande. Sperimentate la tartaruga sulle possibili risposte che vi vengono in mente. Se si desidera avviare un nuovo disegno scrivere i seguenti comandi immediatamente prima del precedente:

\vskip 1cm

\begin{scriptsize}
\begin{minipage}{0.45\textwidth}
\begin{itemize}[itemsep=-3pt,parsep=2pt]
\item[] \hspace{0.5cm} CLEARSCREEN 
\item[] \hspace{0.5cm} HOME
\end{itemize}
\end{minipage}
\end{scriptsize}

\vskip 1cm

Con il primo comando (CLEARSCREEN) si annulla il disegno precedente, con il secondo (HOME) si manda la tartaruga in posizione "home", cioè al centro del foglio con il naso rivolto verso l'alto. Quindi, provate per esempio

\vskip 1cm

\begin{scriptsize}
\begin{minipage}{0.45\textwidth}
\begin{itemize}[itemsep=-3pt,parsep=2pt]
\item[] \hspace{0.5cm} CLEARSCREEN 
\item[] \hspace{0.5cm} HOME
\item[] \hspace{0.5cm} FORWARD 50
\end{itemize}
\end{minipage}
\end{scriptsize}

\vskip 1cm

Di nuovo, premete il pulsante "Run" (Esegui). Cosa è cambiato? 

Ora, per esplorare meglio ciò che la Tartaruga può fare, vi dico un paio di altri comandi: DESTRA e SINISTRA. Non è difficile immaginare a cosa servono questi comandi, non è vero? Ma manca qualcosa, per poterli usare, cosa? Beh, cercate di indovinare e sperimentate. Per favore, riflettete su cosa sta realmente succedendo alla Tartaruga, quando applicate questi comandi. Poi, non rimane che giocare per vedere cosa si può fare. Ricorda, come se foste bambini...

Il motivo per cui insisto così tanto su ciò che i bambini possono fare è che ho avuto diverse opportunità di lavorare con loro e con la tartaruga. Davvero pochi si lamentano della difficoltà, i più semplicemente giocano e si entusiasmano facendo creare al piccolo animale cose divertenti.  Una volta, cercando di disegnare una casa, una bambina tirò fuori una sorta di castello bizzarro. La elogiai commentando quanto fosse meraviglioso questo disegno e lei era così orgogliosa. Poi, quando sono tornato da lei, dopo essermi dedicato alle opere di altri bambini, l'ho trovata in lacrime che singhiozzava disperatamente: aveva accidentalmente cancellato i comandi in un modo che io non riuscii più a recuperare. Mi sono senti molto in colpa per non averle insegnato a salvare le opere di tanto in tanto.

Pertanto, se iniziate a creare qualcosa di più complesso che vi piace, non dimenticate di salvare il documento di tanto in tanto. E' così facile.

In una versione passata del libro questo capitolo era molto più lungo. Dopo l'esperienza che ho accumulato negli ultimi due anni, sia lavorando con gli studenti che con i bambini, preferisco fermarmi qui, invitandovi piuttosto a esplorare liberamente le possibilità. Da qui si può andare in diverse direzioni. Se volete indicazioni ed esempi specifici per usare i comandi di base, o se volete scoprire istruzioni più potenti, le troverete nel prossimo capitolo e in quelli successivi.  Nel capitolo \ref{parte:esperienze-didattiche}, per scoprire fino a che punto si può giungere esplorando, potete leggere la storia di Marta, una mia ex studentessa. Oppure, se volete leggere un modo affascinante di introdurre i bambini al disegno di un cerchio, andate a leggere il capitolo \ref{cap:cerchio}, o meglio, la prima parte di esso, altrimenti comincerete a volare....
 

%La geometria della Tartaruga è una vera geometria, come quella di Euclide o di Cartesio; quella di Euclide è logica, quella di Cartesio è 
%algebrica mentre la geometria della Tartaruga è computazionale. Nella geometria euclidea il punto è l'ente caratterizzato dalla propria posizione e nient'altro, nessuna dimensione, nessuna forma. Per le persone che hanno scarsa familiarità con la matematica il concetto di punto euclideo può sembrare strano, difficile da mettere in relazione con un oggetto della vita reale. Nella geometria della tartaruga il ruolo del punto euclideo è svolto dalla Tartaruga. La Tartaruga è un'immagine molto più intuitiva, anche se presto dovremo metterla in relazione con una posizione - che potremmo identificare con la punta del suo naso, per esempio, ma questo è irrilevante. C'è tuttavia qualcosa di più: un punto ha solo la sua posizione, la Tartaruga ha una posizione ma anche una direzione. È più facile identificarsi con la Tartaruga. Questo dà ai bambini la possibilità di mettere in relazione i concetti matematici formali con la loro esperienza fisica. Vediamo come funziona.

%La Tartaruga comprende alcuni comandi specifici espressi in un linguaggio che Papert chiamava "Turtle talk". Il comando AVANTI fa muovere la Tartaruga in linea retta lungo la direzione in cui è rivolta. Per indicare la distanza da percorrere è necessario aggiungere un numero dopo il comando: FORWARD 1 provoca un piccolo movimento, FORWARD 100 uno più grande. Spesso i bambini sono iniziati a questa logica attraverso una tartaruga meccanica, come la Bee-Bot, che in realtà rappresenta un'ape ma il concetto è lo stesso. Si tratta di un semplice robot programmabile tramite alcuni tasti sul dorso, che riproduce i comandi di base di Turtle talk \footnote{C'è anche il Blue-Bot, un po' più complesso, che ha la possibilità di ricevere una sequenza di comandi da un'applicazione per smartphone tramite una connessione bluetooth}. Vale la pena ricordare che tale versione cibernetica della tartaruga è venuta prima di quella su computer, e fu creata da Papert negli anni Settanta (Fig. 1). Quando i bambini vengono coinvolti in entrambe le attività, con la Bee-Bot (o simile) e con il Logo, sperimentano qualcosa che ha a che fare con un concetto matematico molto potente, Papert direbbe un'idea potente, quello di "isomorfismo": una relazione che coinvolge una corrispondenza biunivoca tra due mondi completamente diversi.

%I comandi FORWARD e BACK fanno sì che la Tartaruga si muova in modo rettilineo lungo il suo percorso: la posizione cambia ma la direzione rimane la stessa. Ci sono anche comandi che cambiano la direzione senza influenzare la posizione: RIGHT e LEFT. Per mezzo di questi comandi la Tartaruga ruota senza cambiare la sua posizione. Per poter lavorare, come nel caso di FORWARD, hanno bisogno di un numero. Per un adulto è facile riconoscere in questi numeri l'angolo di deviazione in gradi. Tuttavia i bambini devono esplorarli e spesso lo fanno con molto divertimento.


