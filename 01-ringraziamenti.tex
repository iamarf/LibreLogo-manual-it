\tableofcontents
\newpage

\section{Ringraziamenti}

In primo luogo ringrazio tutti gli studenti che si sono impegnati oltre le aspettative nel Laboratorio di Tecnologie Informatiche del Corso di Laurea in Scienze della Formazione Primaria, che si è tenuto nell'anno Accademico 2016/2017. L'esplosione di creatività emersa nei liberi esercizi con Logo è stata di grande aiuto nello sviluppo di questo lavoro. 

Poi ci sono stati alcuni contributi particolari, primo fra tutti quello, notevolissimo, di Marta Veloce, i cui "Esercizi di creatività" hanno ispirato il capitolo \ref{cap:marta}. 

Per la seconda parte del medesimo capitolo sono invece debitore a Alberto Averono, insegnante di informatica in un istituto tecnico, che nell'ambito delle attività svolte nel Corso di Perfezionamento "Le competenze Digitali nella Scuola" (2016/2017) ha suggerito delle interessanti variazioni alla proposta di Marta. Uno splendido invito all'impiego verticale di LibreLogo.

Va ringraziata anche la studentessa, della stessa classe di Maria, Eleonora Aiazzi, con il suo testo "Se incontri un professore che ti tratta come un bambino", dove non ha parlato di coding esplicitamente ma ha colto perfettamente il senso del dispositivo didattico che abbiamo tentato di utilizzare. Feedback come questi sono fondamentali per una proposta didattica del genere e rappresentano un contributo importante nella definizione del taglio di un opera come questa.

Abbiamo quindi Antonella Colombo, insegnante di matematica alla scuola primaria, che ci ha regalato una bellissima documentazione di apprendimento sintonico del cerchio, alla Papert. È da questa bella storia che abbiamo presso le mosse per raggiungere alfine la cometa di Halley, nel capitolo \ref{cap:cerchio}.

Un particolare ringraziamento va anche a Giuseppe Albano, un sicuro e raro riferimento per la competenza pedagogica ma anche tecnica. A lui sono grato per il prezioso confronto che mi consente di recuperare riferimenti che avrei faticato a trovare altrimenti, soprattutto per quanto riguarda una fase che definirei d'oro, nella quale ha visto la luce negli anni '70 - '80 un pensiero tecnico-pedagogico del quale si sono, temo, un po' perse le tracce. Un alleato importante insomma. È grazie Giuseppe che ho potuto inserire in bibliografia il testo di Horacio Reggini, Logo: "ali per la mente"\cite{Reggini}, nel quale ho trovato corrispondenze davvero confortanti con il pensiero che sto cercando di promuovere.

Grazie anche all'amico Piero Salonia che mi ha regalato una meticolosa revisione di bozze.

In ultimo grazie a Antonio Fini, "socio" nella conduzione del Laboratorio di Tecnologie Didattiche nella veste di tutor.

Eh... ma sì, grazie al meraviglioso mondo di Linux, che con i suoi "attrezzi" - scp, ssh, rsync, grep, find, nmap, pdflatex, bibtex e via dicendo - mi dona superpoteri ignoti nel mondo dei touchscreen, consentendomi di volare leggiadramente fra router, PC di ogni tipo, server lontani. La vera Internet, la vera libertà. 
