\part{Manuale ragionato di LibreLogo} \label{parte:manuale}

\section{Prefazione}
Questo piccolo manuale nasce per la necessità di fornire supporto di studio e
consultazione nell'insegnamento “Laboratorio di Tecnologie Didattiche” al V
anno del Corso di Laurea Magistrale a ciclo unico “Scienze della Formazione
Primaria” e nell'insegnamento “Laboratorio di Gestione dei Processi Formativi”
al II anno del Corso di Laurea Magistrale “Scienze dell'Educazione degli
Adulti, della Formazione Continua e Scienze Pedagogiche”, presso l'Università
di Firenze, e nell'insegnamento “Informatica” al I anno del Corso di Laurea
Magistrale “Innovazione Educativa e Apprendimento Permanente” presso
l'università telematica Italian University Line. Il manuale guida all'impiego
del linguaggio Logo nella versione LibreLogo implementata all'interno del \textit{word}
\textit{processor} Writer della \textit{suite} di programmi di produttività personale
LibreOffice. LibreLogo è un \textit{plugin} disponibile di \textit{default} in Writer a partire
dalla versione 4.0 di LibreOffice. È stato scritto in linguaggio Python da
László Németh. La documentazione disponibile si trova in http://librelogo.org,
da dove, in particolare, si può scaricare una guida dei comandi di LibreLogo in
italiano \cite{LibreLogo}. Per il resto, sfortunatamente e per quanto è a mia conoscenza sino ad oggi, la documentazione disponibile è tutta in ungherese, principalmente sotto forma di un manuale di esempi scritto dallo stesso László Németh \cite{LibreLogo2} e da un manuale esteso scritto da Lakó Viktória \cite{LibreLogo3}. È a quest'ultimo lavoro che, in una prima fase si è ispirato il presente piccolo manuale, senza tuttavia esserne una traduzione, per vari motivi. In primo luogo io non so l'ungherese e non posso quindi pretendere di poterne fare una vera traduzione e i tempi e le circostanze non mi consentono di avvalermi di un traduttore. Posso tuttavia seguirne le tracce, aiutandomi con i codici (anche se in ungherese quelli si possono imparare), le figure e Google Translate. Del resto, alla fine una traduzione pedissequa non sarebbe nemmeno desiderabile perché viene naturale riformulare il materiale in funzione degli obiettivi specifici e della propria visione della materia. Inoltre, nel corso della traduzione, mi è capitato sempre più spesso di seguire la traccia dei miei pensieri e, alla fine, è stato inevitabile tornare alla fonte primigenia, ovvero al testo con cui Seymour Papert descrisse per la prima volta compiutamente il pensiero che aveva dato origine a Logo, \textit{Mindstorms} \cite{Papert}. È così che ho introdotto la traduzione di due capitoli di \textit{Mindstorms}: il secondo, “\textit{Mathofobia: the Fear of Learning}”, e il terzo, \textit{“Turtle Geometry: A Mathematics Made for Learning”}.

L'immersione profonda nel pensiero di Papert ha poi prodotto un fenomeno interessante. Nei
numerosi passaggi dove Papert insiste sulla necessità di proporre agli studenti nuove idee
matematiche facendo leva sulle conoscenze già possedute (non solo scolastiche) dagli studenti e sul loro coinvolgimento personale, sempre più spesso mi venivano in mente le lezioni di Emma
Castelnuovo, con le quali si impiegano materiali semplici per introdurre tanti concetti matematici. Ad esempio \cite{Castelnuovo}. In questo
libro si riportano alcune lezioni fatte da Emma Castelnuovo presso la Casa-laboratorio di Cenci (Franco Lorenzoni), fra il 2002 e il 2007. La ricerca didattica di Emma Castelnuovo ha riguardato molto l'impiego di materiali semplici per lo studio attivo della matematica.:

\begin{quote}
Ho capito, insomma, che partendo da un materiale semplicissimo (sbarrette, spaghi, elastici ecc.) si potevano costruire i vari capitoli della geometria, motivando i ragazzi a partire da problemi reali. Bastava variare qualche elemento, lasciandone invariati altri, per stimolare delle problematiche anche di alta matematica. Bastava saper guardare attorno a noi perché si aprissero nuove vie del pensiero e si arrivasse, quasi da sé, a formare negli allievi uno spirito matematico.
\end{quote}

Questo pensiero è in accordo completo con quello di Papert. L'unica differenza è costituita dal
contesto nel quale i due autori vanno a ricercare l'interesse e il coinvolgimento degli allievi. Si può dire che la geometria della Tartaruga è un analogo dei materiali fisici usati da Emma Castelnuovo.
Le due visioni e le pratiche che ne scaturiscono non sono affatto in opposizione bensì
complementari. In questa prospettiva, con LOGO si continua e si estende il lavoro
(necessariamente) iniziato con i materiali fisici mantenendo lo stesso identico approccio
pedagogico.

Tutte le figure sono state prodotte con LibreLogo stesso. I codici, adeguatamente commentati, di alcune delle figure sono listati in appendice, come esempio e spunto per ulteriori sviluppi. Nel momento in cui scrivo queste righe ho completato solo il primo capitolo ma trovo utile rendere il lavoro disponibile anche per ricevere eventuali riscontri che potrebbe essere utile per il resto.

\section{Prefazione alla versione 1.0 (settembre 2017)}

L'obiettivo, in questa estate che volge al finire, era quello di iniziare la nuova stagione didattica con tutta la seconda parte completata. Nel frattempo però ho dovuto cambiare completamente il metodo di scrittura, passando, anzi tornando nel mio caso, a \LaTeX, l'unico modo serio di costruire un documento complesso e tipograficamente ineccepibile. E l'unico modo serio per evitare il mal di testa che inevitabilmente coglie chi si azzarda a utilizzare un \textit{word processor} WYSWYG per redigere lavori di una certa dimensione. Questo mi ha obbligato a rimettere mano a tutto il materiale ma ne è valsa la pena. Chi è interessato a approfondire la questione di \LaTeX  può farsi avanti. È un'altra forma di \textit{coding}, se vogliamo, nella stessa logica di HTML. Infatti è un linguaggio di \textit{markup}. non può mancare nel bagaglio di chiunque voglia dedicarsi alle discipline STEM, e forse non solo. 

\section{Prefazione alla versione 1.1 (gennaio 2018)}

Questa versione si differenzia per l'aggiunta della sezione \ref{sez:soluzione-matematica} sulla soluzione matematica del quesito posto da una studentessa, Marta Veloce, intorno al numero di ripetizioni necessarie per la chiusura di cicli di disegno con deviazione totali diverse da 0 o multipli di 360\degree. Marta pose il suo quesito durante la prima edizione del Laboratorio di Tecnologie Didattiche a Scienze della Formazione Primaria il 6 novembre 2016, inviando una decina di pagine di riflessioni in equilibrio fra esplorazione estetica e ragionamento geometrico. Lo scritto si concludeva con la formulazione di una congettura sulla chiusura delle figure geometriche emerse dalla sua esplorazione. Il testo di Marta è riportato nella sezione \ref{sez:esercizi-creativita}

Il mio dovere primario, nel ruolo di professore universitario è quello di rispondere agli studenti. Questo dovere supera quello della ricerca e supera anche la semplice "didattica erogativa", in una scala di valore, perché la domanda difficile di uno studente rappresenta il premio di un percorso dove ricerca e insegnamento hanno innescato una scintilla creativa nella mente di un giovane. Non c'è niente di più alto.

Domande come quella di Marta sono destabilizzanti perché è difficile rispondere. A volte la risposta non c'è. Sono domande vere, domande di ricerca, per tentare di rispondere alle quali occorre onestà intellettuale e umiltà. Discutemmo appronditamente in classe la questione e spiegai subito che la soluzione vera, quella matematica io non l'avevo. Una soluzione matematica è quella che consente di risolvere il quesito in tutte le condizioni possibili. È una soluzione generale. In quella circostanza sviluppai una risposta che consentiva di rispondere al quesito di Marta ma solo nei casi da lei esplorati in una tabella (\ref{tabella-marta}) esposta al termine del suo elaborato. Si trattava cioè di una soluzione euristica, ovvero una soluzione basata su ragionevoli intuizioni ma non ancora sostenuta da un'argomentazione teorica esaustiva. La risposta, ancorché insufficiente, aveva valore didattico perché ci consentiva di mettere a fuoco il significato di verità matematica, tramite il concetto di soluzione euristica. 

Successivamente, durante il corso di perfezionamento “Le competenze digitali nella scuola”, attivato presso il Dipartimento di Scienze della Formazione e Psicologia dell'Università di Firenze nell'anno accademico 2016/2017, sotto la direzione della collega Ranieri, uno dei corsisti, Alberto Averono riprese in mano la questione proponendo una soluzione informatica. L'idea era quella di fornire alla Tartaruga la capacità di riconscere lo stato dal quale era partita in modo da potersi fermare esattamente in quel punto, a partire dal quale avrebbe solo potuto ripetere il percorso fatto. Una soluzione del genere può essere generata per via software, introducendo delle istruzioni che consentano di confrontare lo stato corrente della Tartaruga con quello iniziale. Queste considerazioni hanno consentito di mettere in luce due fatti molto importanti: il concetto di ``stato'' di un sistema, la Tartaruga in questo caso, e la nozione di numero digitale, quale pallida approssimazione dei numeri matematici. Questi fatti sono stati analizzati nella sezione \ref{sez:alternativa-alberto}, In ogni caso, anche se l'approfondimento di Alberto si è rivelato didatticamente assai proficuo, non ci ha fornito la soluzione matematica che desideravamo.

E infine è arrivata anche questa ma è stato necessario approfondire la teoria, cosa che non mi sarei mai aspettato di dover fare in queste circostanze. Ebbene, le questioni affrontate da Marta in sostanza sono quelle che si sono posti Abelson e diSessa\cite{Abelson} nel loro trattato sulla Turtle Geometry. Nella sezione \ref{sez:soluzione-matematica} ho descritto in dettaglio i tratti essenziali della soluzione generale del problema di Marta, fornendo anche il codice per attuarla. La versione 1.1 si distingue dalla 1.0 essenzialmente per questa sezione, a parte altri marginali aggiustamenti.

Un altro aspetto peculiare della versione 1.1 è il fatto che, provvisoriamente, il capitolo \ref{cap:papert2} sia attualmente espunto e disponibile in un file pdf separato e scaricabile da http://iamarf.ch/unifi/Papert-introduce-Logo.pdf. Il motivo sta nel fatto che devo ancora completare il porting verso il sistema di scrittura \LaTeX, di gran lunga più appropriato per la gestione e l'autopubblicazione di questo tipo di testo.

La prossima versione del Piccolo Manuale di Librelogo sarà la 2.0. In questa i capitoli \ref{cap:papert} e \ref{cap:papert2} verranno riuniti e integrati meglio rispetto a alla presente versione. Allo stesso tempo tutto i sorgenti del manuale verranno trasferiti in Github, più adeguato alla gestione di un progetto articolato e complesso.

