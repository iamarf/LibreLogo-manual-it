\chapter{La paura della matematica} \label{cap:papert}

Perché la motivazione fondamentale della genesi di Logo sta tutta nella questione, tutt'ora irrisolta, dell'insegnamento della matematica. Il linguaggio Logo è stato ideato da Papert proprio per cercare di risolvere questo annoso problema per il quale aveva coniato anche un preciso nome, \textit{Mathofobia}, per descrivere la diffusa antipatia verso questa materia. L'interesse di Papert per la questione ha accompagnato tutta la sua vita lavorativa, che si è distesa nella seconda metà del 900. Il suo contributo è stato eccezionale, sia sotto il profilo dell'elaborazione teorica dal punto di vista pedagogico, che della creatività che lo ha condotto a concepire appositamente un linguaggio per avvicinare i ragazzi alla matematica. La profonda competenza, sia nelle questioni matematico-informatiche che in quelle pedagogiche, rende la sua opera unica e spiega la rara capacità di proporre soluzioni concrete. 

La sensazione è che non molto sia cambiato, dagli anni '80 ad ora, almeno in media\footnote{Affermazione che nasconde un mondo di perplessità. Cosa è cambiato? Forse proprio ciò che una “media” non esprime. La scuola cui si riferiva Papert è probabilmente più affine a quella che ha frequentato il sottoscritto (I elementare nel 1960). Allora probabilmente il panorama era più uniforme. “La lo picchi se non capisce perché gliè zuccone!” raccomandò la mamma di un mio compagno di classe alla maestra. I genitori erano alleati di quel sistema scolastico, in una visione formativa che poteva essere coercitiva e punitiva, ma che attraversava tutti i generi di scuole e tutti gli strati sociali. Non c'erano “genitori coach” o “genitori sindacalisti”. Nelle famiglie si lavorava duramente, nelle scuole si faticava. Non c'era ancora il “tempo libero”. La scuola era più brutale, forse iniqua, la pedagogia semplice, ma il panorama era più nitido. Almeno nella provincia rurale degli anni '60 in cui ho vissuto. Ora domina la complessità. Le categorie si intersecano. I dibattiti esplodono, amplificati dai media, a livello microscopico (gruppi di genitori in Wathsapp o Facebook) e a livello macroscopico (stampa, televisione ecc.). Le esperienze personali sono schizofreniche: i miei contatti con il mondo dell'insegnamento rappresentano un quadro affascinante di impegno, studio e sperimentazione; ma le storie private e le narrazioni dei conoscenti sono popolate di pratiche didattiche obsolete e superficiali. La variabilità è allucinante. Dove sta la media? Francamente non sono in grado di valutarlo ma la dispersione è sicuramente molto più ampia di un tempo. A complicare il quadro ci sono le indagini internazionali, paludate di rigore scientifico ma che poi si possono rivelare speratamente fatue. Per alcuni anni è brillata la stella polare della Finlandia nel cielo delle valutazioni PISA dell'OCSE, in particolare per la matematica. Poi emergono una serie di denunce di accademici finlandesi che documentano un crollo delle competenze matematiche: sembra che gli studenti finlandesi siano diventati bravi nei test matematici PISA ma che siano peggiorati in matematica! Leggendo il post di Giorgio Israel “Il bluff della matematica finlandese” (http://gisrael.blogspot.it/2011/05/il-bluff-della-matematica-finlandese.html), che riassume tali denunce, si scopre che i modelli di apprendimento sono banalmente utilitaristici e anche ben lontani dalle idee di Papert che riportiamo qui. Dove sarà la verità? Insomma la confusione regna sovrana e viene seriamente da domandarsi se non ci si debba rassegnare a considerarla inevitabile normalità.}; che la motivazione iniziale, centrata su una seria e difficile rivisitazione del modo di introdurre i giovani alla matematica, sia finita diluita oggi nel calderone del “coding”, nella forma di una sorta di paese dei balocchi, superficialmente entusiasmante per taluni, oggetto di derisione per altri; che il messaggio di Papert, per certi versi estremo e provocatorio, senz'altro da decodificare rispetto ad un'epoca diversa, venga frainteso; che il tutto sia vanificato in sostanza dal fallimento di Logo, rimasto confinato in una minoranza di circoli sperimentali, senza avere rivoluzionato nulla, contrariamente a quelle che sembravano le legittime aspettative di Papert; che invocare la magia della matematica per introdurre i giovani in un dominio comunemente considerato “freddo”, sia un sogno che alla fin fine può concepire solo un matematico, magari un po' idealista, e quindi che solo un'arida via può svelare quella magia, e solo ai pochi in grado di percorrerla, per un motivo o per un altro, e che non possa essere infine altro che così – una cosa che io non voglio pensare ma la paura che sia un po' vera m'è venuta rileggendo \textit{Mindstorms}.

Logo fallito abbiamo detto. Ha fallito nelle intenzioni iniziali di Papert. Non è diffuso nelle scuole e non è adottato come un modo standard per sostenere l'apprendimento della matematica e delle scienze. Ma non ha fallito nel senso di non aver lasciato traccia, al contrario. Ci sono molte versioni del logo in tutto il mondo, alcune delle quali sono diventate importanti strumenti di indagine educativa, ad esempio nel campo della simulazione di sistemi biologici complessi. Ed è sempre da Logo che ha preso spunto il vasto mondo dei linguaggi visivi a blocchi, primo fra tutti Scratch. Logo e Scratch non sono in opposizione. In un certo senso, Scratch deriva da Logo e "contiene" molte delle sue funzionalità. Molte delle cose che puoi fare in Logo possono essere fatte anche in Scratch. Ma Scratch è molto più orientato alla costruzione "del tuo videogioco" o allo storytelling. Il problema, tuttavia, è che questa gamma più ampia di possibilità, in un contesto scolastico caratterizzato da un basso livello di istruzione tecnologica, ha finito per disperdere le intenzioni educative originali di Logo. Una delle intenzioni di questo manuale è quella di recuperare l'originale "sapore matematico" della pratica del \textit{coding} a scuola. Nella prossima sezione commentiamo quella che Papert ha chiamato \textit{mathophobia}, una sorta di malattia che Logo era destinato a combattere. 

\section{\textit{\textit{Mathophobia}: The Fear for Learning}}

Seymour Papert ha scelto di intitolare il secondo capitolo di Mindstorms "Mathophobia: The Fear for Learning". Il suo punto di partenza è la divisione schizofrenica tra scienze umane e scientifiche, una divisione che è profondamente radicata nel linguaggio, nella visione del mondo, nell'organizzazione sociale e nel sistema educativo. Una divisione che negli ultimi 30-40 anni di deriva neoliberista, si è ulteriormente ampliata. Ad esempio, nelle università, a partire dagli anni '80, gli accademici si battono per ottenere i finanziamenti necessari per le loro ricerche. Purtroppo, l'altra faccia della medaglia è che quasi nessuno si preoccupa di insegnare, o molto poco. La carriera di un professore non dipende dalla qualità del suo insegnamento, ma quasi solo dalla quantità della sua produzione scientifica. Le conseguenze sono pessime. Gli accademici tendono a trasformarsi in manager quando fanno ricerca e in funzionari pubblici quando insegnano: imprenditori dinamici da un lato e (forti) conservatori dall'altro. In questo modo, anche nelle scienze umane si assiste a una deriva tecnica del ruolo accademico. La missione dell'insegnamento, che in un certo senso è il "lato umanistico" del lavoro, si riduce a una sorta di Cenerentola. Così, la dicotomizzazione tra scientifico e umanistico è ancora più forte e sbilanciata.

La questione non è quella di un equilibrio adeguato, bensì di spezzare la linea di demarcazione tra le due culture. Papert guardò il computer come una forza per attenuare la distinzione. La pratica del \textit{coding} è stata pensata come un modo per introdurre una matematica più umanistica e per sfruttare elementi di pensiero scientifico nelle scienze umane. È in questo contesto che Papert ha parlato di una \textit{Mathland}, dove la matematica costituirebbe un vocabolario naturale, con l'idea che potremmo cambiare non solo il modo in cui insegniamo la matematica, ma anche il modo in cui la nostra cultura concepisce la conoscenza e l'apprendimento.

Papert sostiene che le sue argomentazioni non si limitano all'apprendimento della matematica, ma riguardano anche l'atteggiamento nei confronti dell'apprendimento in generale. La parola \textit{mathphobia} suggerisce due associazioni. Una è il diffuso timore per la matematica. L'altra deriva dalla radice "math", che in greco antico significa apprendimento in senso generale. Così, se i bambini iniziano ad essere abili studenti spontanei, in seguito "imparano" la paura di imparare qualsiasi cosa e non solo la matematica. Ironia della sorte, sembra che più si studia, più si ha paura di imparare.

I bambini imparano migliaia di parole prima di entrare in prima elementare. Meno evidente per molte persone è il fatto che i bambini imparano una grande quantità di matematica pure. Tra queste conoscenze prescolare ci sono per esempio nozioni come la conservazione del volume di liquidi in recipienti di forma diversa, o l'indipendenza del numero totale di oggetti dall'ordine con il quale sono stati contati. Papert ha chiamato questo 

\begin{quote}
apprendimento Piagetiano, un processo di apprendimento che ha molte caratteristiche che le scuole dovrebbero invidiare: è efficace (tutti i bambini ci arrivano), è economico (sembra non richiedere né insegnanti né uno sviluppo del curriculum), ed è umano (i bambini sembrano farlo in uno spirito spensierato, senza esplicita ricompensa esterna e punizione).
\end{quote}.  

Infatti, le pratiche dell'insegnamento della matematica sottovalutano in larga misura l'apprendimento piagetiano, imponendo conoscenze formali che risultano per lo più dissociate dalle precedenti nozioni spontanee. Le conseguenze per i futuri adulti sono pesanti. La perdita dell'atteggiamento positivo del bambino nei confronti dell'apprendimento è un fenomeno molto comune nell'età adulta e non riguarda solo la matematica:

\begin{quote}
La deficienza diventa identità: "Non posso imparare il francese, non ho orecchio per le lingue"; "Non potrei mai essere un uomo d'affari, non ho una testa per le cifre".
\end{quote}.  

Circa l'80\% dei miei studenti di scuola primaria si dichiara "distante dalla matematica". Molti sostengono di avere difficoltà anche con le tecnologie, nonostante si supponga che siano "nativi digitali".

L'idea che ci siano persone intelligenti e stupide ai fini di una data attività è diffusa. È estremamente difficile sradicare i pregiudizi sui propri atteggiamenti. Il punto è che le convinzioni accettate circa l'attitudine matematica non sono dimostrabili. Per rafforzare il concetto, Papert ha riformulato l'argomento come segue \cite{Papert} (p. 43):

\begin{quote}
Immaginiamo di far disegnare ai bambini per un'ora al giorno passi di danza sulla carta e di far sostenere loro un esame su tali “questioni di danza” prima di lasciarli ballare veramente. Non dovremmo in tal caso aspettarci un mondo pieno di “danzofobi”? E non concluderemmo che coloro che ce la fanno a raggiungere la sala da ballo sono i più dotati per la danza? Io credo che sia altrettanto ingiustificato trarre conclusioni  sulle doti matematiche in base allo scarso entusiasmo dei bambini per passare centinaia di ore a fare somme.
\end{quote}.

Spesso è la scuola che costruisce le attitudini:

\begin{quote}
Consideriamo il caso di un bambino  che ho seguito durante i suoi ottavo e nono anni di età. Jim era un bambino molto loquace ma matofobico appartenente ad una famiglia di professionisti. La sua passione per le parole e il piacere di parlare si erano manifestate molto prima di andare a scuola. La matofobia era invece comparsa a scuola. La mia teoria è che essa sia stata una diretta conseguenza della sua precocità verbale. Dai genitori avevo appreso che Jim aveva presto sviluppato l'abitudine di commentare a voce alta qualsiasi cosa facesse. Un'abitudine che non aveva causato particolari problemi con i genitori o presso la scuola dell'infanzia. I problemi sono sorti affrontando l'aritmetica. A quel punto aveva già imparato a tenere sotto controllo la sua abitudine ma io sospetto che lui non avesse cessato di commentare le proprie azioni, seppur interiormente. Durante le ore di matematica si trovava in imbarazzo: semplicemente non riusciva a commentare l'attività di fare somme. Gli mancava il vocabolario (come manca alla maggior parte di noi) e non vedeva la motivazione. Questa frustrazione si tramutò in odio per la matematica, la conseguenza del quale fu una valutazione di scarsa attitudine per la materia. 

Per me fu una storia commovente. Credo che molto spesso quella che appare una debolezza intellettuale sia espressione, come nel caso di Jim, di quella che in realtà è una particolare capacità. È non è solo la capacità verbale, chiunque osservi con attenzione i bambini nota processi simili: per esempio un bambino che prediliga l'ordine logico può avere problemi con la sillabazione dell'inglese e magari finire col detestare la scrittura.
\end{quote}.

L'idea di Papert era che si possono usare i computer come veicoli per risolvere situazioni come quella di Jim o quella di bambini che, all'opposto, amano la logica, ma presentano problemi di tipo dislessico. Entrambi i casi sono vittime della netta separazione della nostra cultura tra "il verbale" e "il matematico". Egli immagina una sorta \textit{Mathland} dove l'amore di Jim per la lingua e la sua abilità potrebbero essere mobilitati per realizzare l'apprendimento matematico formale, invece di contrastarlo, mentre per l'altro tipo di bambini, l'amore per la logica potrebbe nutrire l'interesse per la linguistica. I metodi di insegnamento prevalenti danno agli studenti di matematica possibilità limitate di reperire un senso in ciò che stanno imparando. Di conseguenza, i bambini sono costretti a seguire il modello peggiore per l'apprendimento della matematica, che è l'apprendimento mnemonico, dove l'oggetto di studio sembra non avere senso. È quello che Papert chiamava un modello di apprendimento \textit{dissociato}.

\begin{quote}
Nel corso di uno studio di un anno, in una classe II di scuola  media di I grado di livello medio, una delle attività era quella che gli studenti chiamavano “\textit{computer} \textit{poetry}”. L'attività consisteva nell'usare il computer per comporre frasi: loro inserivano una struttura sintattica che il computer popolava di parole in maniera casuale. Il risultato è una sorta di poesia concreta tipo quella illustrata qui sotto\footnote{[NdR] Ho lasciato la versione originale, mi pare inutile “tradurre” un pezzo simile, ai fini della comprensione del concetto. Ho riportato questo esempio perché verrà ripreso, in altra forma, in uno degli esempi più avanzati di impiego di Logo. É interessante che ciò che negli anni 70 richiedeva un potente computer e uno staff di ricerca avanzato ora si possa fare con una semplice implementazione di Logo in un PC.}:
\end{quote}

\begin{quote}
INSANE RETARD MAKES BECAUSE SWEET SNOOPY SCREAMS\\
SEXY GIRL LOVES THATS WHY THE SEXY LADY HATES\\
UGLY MAN LOVES BECAUSE UGLY DOG HATES\\
MAD WOLF HATES BECAUSE INSANE WOLF SKIPS\\
SEXY RETARD SCREAMS THATS WHY THE SEXY RETARD\\
THIN SNOOPY RUNS BECAUSE FAT WOLF HOPS\\
SWEET FOGINY SKIPS A FAT LADY RUNS\\
\end{quote}

\begin{quote}
Un'allieva di tredici anni, Jenny, aveva commosso lo staff del progetto chiedendo il primo giorno: “Perché avete scelto noi? Noi non siamo i cervelloni. (“\textit{Why were we chosen for this? We're not the brains.}”. Lo studio prevedeva proprio di lavorare con una classe di livello “medio”. Un giorno Jenny entrò tutta eccitata. Aveva fatto una scoperta: “Ora ho capito perché ci  sono i sostantivi e i verbi.” Già da vari anni Jenny aveva fatto esercizi grammaticali, ma non aveva mai capito le differenze fra sostantivi, verbi e avverbi. Ma ora era chiaro che le sue difficoltà non dipendevano dall'incapacità di lavorare con categorie logiche. Il problema era un altro. Lei non aveva semplicemente compreso la finalità della fatica. Non era stata in grado di afferrare il senso della grammatica perché non vedeva a cosa servisse. E quando aveva chiesto a cosa serviva, la spiegazione dell'insegnante le era parsa manifestamente disonesta: “La grammatica ti serve a parlare meglio.”

Infatti, per recuperare la connessione fra l'apprendimento della grammatica e il miglioramento della lingua parlata occorre una visione più ampia del complesso procedimento di apprendimento di una lingua, che Jenny non poteva avere all'età in cui era entrata in contatto con la grammatica. Certamente lei non poteva vedere in che modo la grammatica potesse aiutarla a parlare meglio, né pensava di avere necessità di essere aiutata. Di conseguenza aveva sviluppato un sentimento di rancore per la grammatica. E, come succede alla maggior parte di noi, il rancore garantisce fallimento. Ma quando si è trovata nella condizione di far comporre frasi al computer, è successo qualcosa di interessante, trovandosi nella condizione di dover classificare le parole in categorie non perché qualcuno le avesse chiesto di farlo ma perché ne aveva bisogno. Per “insegnare” al suo computer come comporre serie di parole in maniera che sembrassero frasi compiute occorreva “insegnargli” a scegliere parole appartenenti alle categorie giuste. Ciò che lei aveva imparato sulla grammatica tramite l'esperienza con una  macchina non aveva niente di meccanico né di routinario. Il suo era stato un apprendimento profondo e significativo. Jenny aveva fatto più che imparare le definizioni per una particolare classe grammaticale. Aveva capito l'idea generale che le parole (come le cose) possono essere collocate in gruppi o insiemi diversi, e che fare questo può essere utile. Non aveva solo “capito” la grammatica ma aveva cambiato il suo atteggiamento nei suoi confronti. Era “sua”, e nel corso dell'anno, altri casi simili l'aiutarono rivedere la propria immagine. Cambiarono anche i suoi risultati; i suoi voti, prima medio-bassi, divennero massimi per il resto degli anni scolastici. Imparò che anche  lei poteva essere “un cervellone”, dopo tutto.
\end{quote}

Papert è perentorio: spesso i bambini non riescono a capire a cosa servano realmente matematica e grammatica perché percepiscono quelle degli adulti come spiegazioni volutamente ambigue.

\begin{quote}
È naturale come matematica e grammatica non vengano capite dai bambini quando non sono capite da chi sta loro intorno e come, affinché la comprendano, occorra qualcosa di più di un insegnante che dica la cosa giusta o disegni il diagramma giusto alla lavagna. Ho chiesto a molti insegnanti e genitori cosa pensassero della matematica e perché fosse importante impararla. Pochi di loro hanno espresso una visione sufficientemente coerente da giustificare l'impiego di varie migliaia di ore della vita di un bambino per impararla, e questo i bambini lo sentono. Quando un insegnante spiega a uno studente che tutte quelle ore di aritmetica servono a essere in grado di controllare il resto al supermercato, questo non viene semplicemente creduto. I bambini interpretano tali “motivazioni” come un ulteriore esempio di malafede da parte degli adulti. Lo stesso effetto si manifesta dicendo ai bambini che la matematica scolastica è “divertente”, quando è loro chiaro che gli insegnanti che si esprimono così per divertirsi fanno tutt'altre cose. Ne aiuta molto spiegare che la matematica serve per diventare scienziati poiché la maggior parte di loro non prevede una cosa del genere.  La maggior parte dei bambini si rende conto che l'insegnante non ama la matematica più di quanto la amino loro e che la ragione per cui va fatta è semplicemente perché lo prevede il curricolo. Tutto ciò erode la fiducia dei bambini nel mondo degli adulti e nel processo di educazione. \textit{E io penso che introduca un elemento di profonda disonestà nella relazione educativa}\footnote{[NdR] Corsivo dell'autore.}.
\end{quote}

È importante tenere presente la differenza tra la matematica - un vasto campo di indagine la cui bellezza è raramente sospettata dalla maggior parte dei non addetti ai lavori - e la \textit{matematica scolastica}. Quest'ultima è una sorta di costruzione sociale, cioè un insieme di argomenti matematici determinati da una successione di circostanze specifiche. Un processo che non garantisce, \textit{per se}, il raggiungimento di un risultato ottimale. Ricorda la storia del layout della tastiera noto con la sigla QWERTY, che rappresenta i primi cinque tasti delle righe superiori. Questo accordo non ha una spiegazione razionale, ma solo storica. Fu introdotto perché i tasti delle prime macchine da scrivere tendevano ad incepparsi. Furono disposti così per ridurre le collisioni, separando i tasti che si susseguivano più frequentemente. La tecnologia delle macchine da scrivere migliorò rapidamente e in pochi anni il problema dell'inceppamento non fu più un problema ma la disposizione QWERTY rimase tal quale. A questo punto troppe persone conoscevano il layout QWERTY e la produzione di macchine da scrivere era troppo lontana per fare un passo indietro nella riprogettazione di un layout più razionale, ad esempio raggruppando i tasti più usati. 

La questione della tastiera QWERTY è un buon esempio di come le abitudini consolidate non rappresentino necessariamente la scelta migliore. Anche la matematica scolastica si è formata in un contesto storico diverso. Allo stesso modo questa idea di matematica si è consolidata profondamente e, ancora oggi, per la maggior parte delle persone è inconcepibile che la matematica possa essere anche qualcos'altro. Ricordo un noto professore di analisi matematica che, all'inizio del primo anno del corso di laurea in matematica, esortava i suoi studenti a dimenticare ciò che avevano imparato al liceo, poiché la matematica era un'altra cosa. 

La geometria della tartaruga è stata concepita per adattarsi ai bambini e, prima di tutto, per essere "appropriabile". Potremmo descrivere questo concetto attraverso alcuni principi. In primo luogo, il principio di continuità: le nuove conoscenze matematiche devono essere in continuità con quelle esistenti, quelle che i bambini hanno prima di andare a scuola. Poi il principio di potenza: la nuova conoscenza deve permettere agli studenti di realizzare progetti personali significativi, che non potrebbero essere realizzati senza di essa. Infine, il principio della risonanza culturale: i nuovi concetti devono avere un senso per i bambini nel loro contesto sociale; paradossalmente, anche nel contesto sociale degli adulti: non dobbiamo infliggere ai bambini qualcosa che non abbiamo compreso a fondo e, purtroppo, spesso con la matematica scolastica è proprio così. 
